\documentclass[11pt]{article}
\usepackage{sigsam, amsmath, sagetex, tikz}

% PAGE LIMIT: FOUR PAGES!

% leave as is
\issue{Vol.xx, No.xx, Issue xxx, month 201x}
\articlehead{ISSAC 2019 abstracts}
\titlehead{Multivariate Ore Polynomials in Sage}
\authorhead{Manuel Kauers, Marc Mezzarobba}
%\titlehead{Title of your paper}
%\authorhead{Author's Name}
\setcounter{page}{1}

\def\<#1>{\langle#1\rangle}
\let\Bold\mathbb
\let\set\mathbb

\begin{document}

\title{Multivariate Ore Polynomials in Sage}

\author{%
  Manuel Kauers\footnote{Supported by the Austrian FWF grants P31571-N32, F5004} \
  and Marc Mezzarobba\footnote{??}\\[\medskipamount]
  Institute for Algebra $\cdot$ Johannes Kepler University $\cdot$ Linz, Austria\\
  CNRS, LIB6 $\cdot$ Sorbonne Universit\'e $\cdot$ Paris, France\\[\medskipamount]
  \url{manuel@kauers.de} $\cdot$ \url{marc@mezzarobba.net}
}

\date{}

\maketitle

\begin{abstract}
  We present the latest update of the ore\_algebra package for Sage.
  The main new feature in this release is the support of operators in several variables.
\end{abstract}

\section{Introduction}

Ore polynomials are important features of computer algebra systems because they provide
functionality for working with differential and recurrence operators that describe
so-called D-finite functions and sequences~\cite{stanley80,kauers10j,kauers13}. Maple has
the OreTools package~\cite{abramov03} and gfun~\cite{salvy94}, for Mathematica we have
the package of Koutschan~\cite{koutschan10c},
and there is the ore\_algebra package by Kauers, Jaroschek and Johansson~\cite{kauers14b} for
Sage~\cite{zimmermann18}. This software demo abstract is about the most recent version of the
latter package.

Earlier versions of the ore\_algebra package already provided functionality for basic arithmetic
and actions; GCRD and LCLM; D-finite closure properties; natural transformations between related
algebras; guessing; desingularization; solvers for polynomials, rational functions and (generalized)
power series. Mezzarobba~\cite{mezzarobba16} added a subpackage ore\_algebra.analytic for arbitrary-precision
evaluation of D-finite functions, which is now fully integrated into the main version of the package.
Hofstadler~\cite{hofstadler19} contributed a set of convenience functions which support a better integration
with symbolic expressions.

Almost all the features so far were limited to the univariate case. The main contribution in the
latest version is an extension to the multivariate case. While basic arithmetic of Ore polynomials
in several variables has been supported since the beginning, we now also offer implementations of
D-finite closure properties and creative telescoping as well as a (rudimentary) implementation of
Gr\"obner bases. 

\section{Basic Examples}

After loading the package and creating the base ring $\set Z[x,y]$, the Ore algebra of differential
operators for $x$ and $y$ can be created as follows. 
\begin{sageexample}
  sage: from ore_algebra import *
  sage: R.<x,y> = PolynomialRing(ZZ)
  sage: A.<Dx,Dy> = OreAlgebra(R)
  sage: A
\end{sageexample}
The usual arithmetic respects the required commutation rules. 
\begin{sageexample}
  sage: (Dx + Dy)*(y + x)
  sage: (x + y)*(Dx + Dy)
\end{sageexample}
Differential operators can be applied to objects which Sage knows to differentiate and to multiply
by base ring elements:
\begin{sageexample}
  sage: (Dx + Dy)(sin(x*y))
\end{sageexample}
Using $Sx,Sy$ instead of $Dx,Dy$ creates an Ore algebra with shift operators instead of differential operators.
Of course, also combinations like $\set Z[x,y]\<Sx,Dy>$ and other generators besides shift and derivation are possible.
The available operators are the same as in the univariate case, see~\cite{kauers14b} for more details.

In the univariate case, methods for executing closure properties were directly attached to the annihilating operators.
In the multivariate case, where several annihilating operators are needed to describe a D-finite object, we must work
with left ideals of Ore algebras. They can be constructed as follows.
\begin{sageexample}
  sage: I = A.ideal([Dy^2, Dx*Dy, Dx^2 - y*Dy + 1])
  sage: I
\end{sageexample}
Note that the base ring is always converted to a field.

Various ideal theoretic operations are available as methods. 
\begin{sageexample}
  sage: I.eliminate([Dy])
  sage: Dx^3 + Dx in I # (membership)
  sage: J = A.ideal([-Dy^2 + x*Dx - 1, Dx*Dy, Dx^2])
  sage: I == J
  sage: K = I.intersection(J) # closure property `addition`
  sage: K <= J # (inclusion)
  sage: K = I.symmetric_product(I) # closure property `multiplication`
  sage: K.vector_space_basis() # terms under the staircase
  sage: I.annihilator_of_associate(Dx + Dy) # closure property `ore action`
  sage: I.annihilator_of_composition(x=x^2,y=1-x^2) # closure property `composition`
\end{sageexample}
Most of these operations rely on a generic FGLM implementation which uses the packages own linear system solvers.
There is also a direct implementation of Buchberger's algorithm for computing Gr\"obner bases, which however is
not highly optimized. 
\begin{sageexample}
  sage: I.groebner_basis()
\end{sageexample}
All operations discussed so far are generalizations of the univariate counterparts that already existed
in earlier versions. A feature that is inherently multivariate is creative telescoping~\cite{zeilberger91,chyzak00,chyzak14}.
Here, the input is a left ideal $I\subseteq A$, say $A=\set Q(x,y)\<Dx,Dy>$ and the task is to find
the ideal consisting of all operators $P\in\set Q(x)\<Dx>$ such that there exists $Q\in A$ with $P-D_yQ\in I$.
The operators $P$ are called \emph{telescopers,} and the corresponding operators $Q$ are called their \emph{certificates.}
Creative telescoping is a central operation for evaluating definite sums and integrals~\cite{chyzak14}.
We cannot explain here in detail the relevance of creative telescoping for these applications, but the use of the
package is easily demonstrated:
\begin{sageexample}
  sage: A.ideal([Dx - 2*x*y^2, Dy - 2*x^2*y]).ct(Dy)
\end{sageexample}
The output is a pair $([P_1,P_2,\dots],[Q_1,Q_2,\dots])$ whose first component is a basis of the ideal of telescopers
and whose second component contains the certificates corresponding to the basis elements.
The example above originates from the integration problem $\int_y\exp(x^2y^2)dy$. The following example is the essence
of a proof of the binomial theorem $\sum_k\binom nk=2^n$:
\begin{sageexample}
  sage: B.<Sx,Sy> = OreAlgebra(R)
  sage: B.ideal([(y+1)*Sy + (y-x), (x-y+1)*Sx - (x+1)]).ct(Sy - 1)
\end{sageexample}
Finally, also the guessing features have been extended to the multivariate setting. For example, annihilating operators
for the binomial coefficients can be found as follows.
\begin{sageexample}
  sage: data = [[binomial(n,k) for k in range(20)] for n in range(20)]
  sage: guess(data, B, point_filter = lambda n,k : k<=n).groebner_basis()
\end{sageexample}
The option \verb|point_filter| informs the guessing engine which part of the input array contains interesting data.
A number of further options is specified in the documentation. 

\section{Some Slightly Larger Examples}

The examples shown above need virtually no computation time. In order to get some idea about the performance of our
implementation, we consider a set of 19 creative telescoping problems that appeared in a study of restricted lattice
walks~\cite{bostan16b}.
Each of the 19 problems starts from a certain rational function $r\in\set Q(u,v,t)$ and the task consists of two
applications of creative telescoping. Writing $I$ for the ideal of annihilating operators of $r$ in $\set Q(u,v,t)\<Du,Dv,Dt>$,
we first have to compute the ideal $T_v$ in $\set Q(u,t)\<Du,Dt>$ of all telescopers of $I$ with respect to~$Dv$ and then
the ideal $T_{u,v}$ in $\set Q(t)\<Dt>$ of telescopers of $T_v$ with respect to~$Du$. The smallest rational function in
the collection is
\[
\tfrac{-u^2 v^2+u^2+v^2-1}{t u^3 v^2-t u^3 v+t u^2 v^3-2 t u^2 v^2+2 t u^2 v-t u^2-t u v^3+2 t u v^2-2 t
  u v+t u-t v^2+t v-u^2 v^2+u^2 v+u v^2-u v},
\]
and the largest is about twice as long. The complete data is available in ore\_algebra.examples.smallsteps.

In the table below, we compare our performance to that of Koutschan's package~\cite{koutschan10c}. We do not claim that a
performance comparison will look similarly for any creative telescoping problem, but we think the experiment is at least
an indication that our implementation is not too bad.

\begin{center}
  % timings taken on picard.algebra.uni-linz.ac.at
  % with Sage 8.7 and ore_algebra version 7ab0f7e821cde38e1ace6ae0844bc34e726aaf5a
  % with Mathematica 12 and HolonomicFunctions.m version 1.7.3
  % on 2019-04-25
  \begin{tabular}{c|ccccccccccccccccccc}
    case & 1 & 2 & 3 & 4 & 5 & 6 & 7 & 8 & 9  \\\hline
    ore\_algebra & 601ms & 668ms & 705ms & 862ms & 4.82s & 10.3s & 5.22s & 10.7s & 11.1s \\
    HolonomicFunctions.m & 1.55s & 1.59s & 1.86s & 2.09s & 25.2s & 51.9s & 26.8s & 55.1s & 51.5s
  \end{tabular}

  \begin{tabular}{c|ccccccccccccccccccc}
    case & 10 & 11 & 12 & 13 & 14 & 15 & 16 & 17 & 18 & 19 \\\hline
    ore\_algebra & 11.9s & 4.46s & 8.57s & 11.1s & 11.7s & 4.21s & 8.27s & 18.9s & 36.2s & 2.05s \\
    HolonomicFunctions.m & 63.4s & 19.4s & 35.4s & 51.3s & 62.3s & 19.1s & 34.7s & 198s & 445s & 5.77s
  \end{tabular}
\end{center}

\small
\bibliographystyle{plain}
\bibliography{bib}

\end{document}
